% ~~~~~~~~~~~~~~~~~~~~~~~~~~~~~~~~~~~~~~~~~~~~~~~~~~~~~~~~~~~~~~~~~~~~~~~~~~~~~
%                                   ABSTRACT
% ~~~~~~~~~~~~~~~~~~~~~~~~~~~~~~~~~~~~~~~~~~~~~~~~~~~~~~~~~~~~~~~~~~~~~~~~~~~~~
\vspace*{0.5cm}
\noindent
{\textbf{\Large{Abstract}}}
\vspace*{0.5cm}
\normalsize \\
Quoridor is a two/four-player competitive game whose objective is to reach
opposite side of gameboard sooner than opponents. Till today no satisfying
artificial intelligent system exists that could challenge advanced players.
In this work, we summarize existing strategies for designing intelligent agents
and propose connectionist-based solution with combination of reinforcement
learning. Considering no dataset of expert {human-player} games exists we have designed
simple heuristic to act as a teacher in supervised learning. Our agent has been
able to successfully imitate behaviour of heuristic player which offers
potential for mastering the game providing expert-game dataset.
\clearpage

\vspace*{0.5cm}
\noindent
{\textbf{\Large{Abstrakt}}}
\vspace*{0.5cm}
\normalsize \\
Quoridor je kompetitívna hra dvoch/štyroch hráčov, ktorých cieľom je dostať sa
na opačnú stranu hernej dosky skôr ako súperi. Do dnešného dňa stále neexistuje
prijateľná umelá inteligencia, ktorá by bola výzvou pre skúsených hráčov.
V tejto práci sumarizujeme existujúce stratégie pre návrh inteligentných
agentov a navrhujeme konekcionistické riešenie v kombinácii s učením
posilňovaním. Pretože neexistuje žiadna databáza odohratých hier skúsených
hráčov, navrhli sme jednoduchú heuristiku, ktorá slúžila ako učiteľ
pri trénovaní. Náš agent sa dokázal úspešne naučiť imitovať správanie
heuristického hráča, čo ponúka potenciál, aby sa naučil hrať ako expert
v prípade poskytnutí databázy hier skúsených hráčov.
\clearpage
