% ~~~~~~~~~~~~~~~~~~~~~~~~~~~~~~~~~~~~~~~~~~~~~~~~~~~~~~~~~~~~~~~~~~~~~~~~~~~~~
%                                 INTRODUCTION
% ~~~~~~~~~~~~~~~~~~~~~~~~~~~~~~~~~~~~~~~~~~~~~~~~~~~~~~~~~~~~~~~~~~~~~~~~~~~~~
\chapter{Introduction}\label{chap:1}
  \lhead{Chapter 1. \emph{Introduction}}
% 1. we play games because it is in our nature, it helps us to improve
% 2. one such game helping us to improve is Quoridor with these rules
% 3. there are many ways to solve games, such as QL or ANN
% 4. no big success even with ANN
% 5. i will try to implement ANN with QLNN

Hrame hry, lebo nas to bavi. Chceme silnych superov, aby sme sa sami mohli
zlepsovat. Je vela hier, ktorymi mozme rozvijat svoje mentalne schopnosti.
Pre niektore z nich vsak uz boli najdene optimalne strategie a teda nie su
vyzvou.
Avsak, hra Quoridor zatial medzi ne nepatri. Kedze bola vymyslena len
pomerne nedavno, aj pokusov o vytvorenie pocitacovych agentov bolo poskromne.
Dokonca, zatial neboli ani uspechy s pouzitim neuronovych sieti.
Toto je presne to, co sa vramci tejto prace pokusim zmenit. Skombinujem
metodu ucenia s posilnovanim, tzv. Q-learning spolu so schopnostami neuronovej
siete zovseobecnovat.

TODO
