% ~~~~~~~~~~~~~~~~~~~~~~~~~~~~~~~~~~~~~~~~~~~~~~~~~~~~~~~~~~~~~~~~~~~~~~~~~~~~~
%                                 INTRODUCTION
% ~~~~~~~~~~~~~~~~~~~~~~~~~~~~~~~~~~~~~~~~~~~~~~~~~~~~~~~~~~~~~~~~~~~~~~~~~~~~~
\chapter{Introduction}\label{chap:1}
  \lhead{Chapter 1. \emph{Introduction}}
% 1. we play games because it is in our nature, it helps us to improve
% 2. one such game helping us to improve is Quoridor with these rules
% 3. there are many ways to solve games, such as QL or ANN
% 4. no big success even with ANN
% 5. i will try to implement ANN with QLNN

Since it is in our nature to play games, we seek for better opponents to
improve ourselves.
There are variety of games where we can enhance our mental abilities and many
of them are well known for centuries.

However, there were found optimal strategies for some games, which may
sometimes seem either less challenging or even less interresting.
Nevertheless, game Quoridor is not among these yet.
Considering it has been invented relatively recently (1997), attempts realized
to create a computer agent have been scarce.
Additionaly, there has been no success with the use of artifitial neural
networks.  Also, game complexity estimation created by mertens has not been
properly addressed to the authors knowledge.

In this work, we will make brief overview of existing approaches to the
production of agents playing games in the background section.
Game itself with history will be described in latter section together with
closer upper bound complexity estimation.

Within the scope of this thesis, we have implemented a simple
console application where Quoridor can be played.
As part of this application we have tried implementing and training neural
network, for which an opponent with simple heuristics will be created as
an reference point to the improvement of the trained network.
Moreover, we will try to combine reinforcement learning method 'Q-learning'
with networks ability to generalize to estimate the best action.

In the last section we demonstrate results achieved.
