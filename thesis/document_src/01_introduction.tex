% ~~~~~~~~~~~~~~~~~~~~~~~~~~~~~~~~~~~~~~~~~~~~~~~~~~~~~~~~~~~~~~~~~~~~~~~~~~~~~
%                                 INTRODUCTION
% ~~~~~~~~~~~~~~~~~~~~~~~~~~~~~~~~~~~~~~~~~~~~~~~~~~~~~~~~~~~~~~~~~~~~~~~~~~~~~
\chapter{Introduction}\label{chap:1}
  \lhead{Chapter 1. \emph{Introduction}}
% 1. we play games because it is in our nature, it helps us to improve
% 2. one such game helping us to improve is Quoridor with these rules
% 3. there are many ways to solve games, such as QL or ANN
% 4. no big success even with ANN
% 5. i will try to implement ANN with QLNN

Since it is in our nature to play games, we seek for better opponents to
improve ourselves.
There are variety of games where we can enhance our mental abilities and many
of them are well known for centuries.

However, there were found optimal strategies for some games, which may
sometimes seem eigther less challenging or even less interresting.
Nevertheless, game Quoridor is not among these yet.
Considering it has been invented relatively recently (1997), attempts realized
to create a computer agent have been scarce.
Additionaly, there has been no success with the use of artifitial neural
networks.  Also, commonly used game complexity estimate has not been reviewed
to the authors knowledge.

First, brief overview of existing approaches to the production of agents
playing games will be presented in the background section.
Game itself with history will be described in latter section, where game
complexity reestimate will be present.

Also within the scope of this thesis, implemented will be a simple
console application where Quoridor can be played.
As part of this application I will try to implement and train neural network,
for which an opponent with simple heuristics will be created as an reference
point to the improvement of the trained network.
Moreover, I will try to combine reinforcement learning method 'Q-learning'
with networks ability to generalize to estimate the best action.

Presentation with gathered results during the training will follow.
